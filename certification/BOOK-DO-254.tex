\documentclass[]{article}
\usepackage{lmodern}
\usepackage{amssymb,amsmath}
\usepackage{ifxetex,ifluatex}
\usepackage{fixltx2e} % provides \textsubscript
\ifnum 0\ifxetex 1\fi\ifluatex 1\fi=0 % if pdftex
  \usepackage[T1]{fontenc}
  \usepackage[utf8]{inputenc}
\else % if luatex or xelatex
  \ifxetex
    \usepackage{mathspec}
  \else
    \usepackage{fontspec}
  \fi
  \defaultfontfeatures{Ligatures=TeX,Scale=MatchLowercase}
\fi
% use upquote if available, for straight quotes in verbatim environments
\IfFileExists{upquote.sty}{\usepackage{upquote}}{}
% use microtype if available
\IfFileExists{microtype.sty}{%
\usepackage[]{microtype}
\UseMicrotypeSet[protrusion]{basicmath} % disable protrusion for tt fonts
}{}
\PassOptionsToPackage{hyphens}{url} % url is loaded by hyperref
\usepackage[unicode=true]{hyperref}
\hypersetup{
            pdftitle={Advanced IP},
            pdfauthor={QueenField},
            pdfborder={0 0 0},
            breaklinks=true}
\urlstyle{same}  % don't use monospace font for urls
\usepackage[left = 3cm, right = 2cm, top = 3cm, bottom = 2cm]{geometry}
\usepackage{graphicx,grffile}
\makeatletter
\def\maxwidth{\ifdim\Gin@nat@width>\linewidth\linewidth\else\Gin@nat@width\fi}
\def\maxheight{\ifdim\Gin@nat@height>\textheight\textheight\else\Gin@nat@height\fi}
\makeatother
% Scale images if necessary, so that they will not overflow the page
% margins by default, and it is still possible to overwrite the defaults
% using explicit options in \includegraphics[width, height, ...]{}
\setkeys{Gin}{width=\maxwidth,height=\maxheight,keepaspectratio}
\IfFileExists{parskip.sty}{%
\usepackage{parskip}
}{% else
\setlength{\parindent}{0pt}
\setlength{\parskip}{6pt plus 2pt minus 1pt}
}
\setlength{\emergencystretch}{3em}  % prevent overfull lines
\providecommand{\tightlist}{%
  \setlength{\itemsep}{0pt}\setlength{\parskip}{0pt}}
\setcounter{secnumdepth}{0}
% Redefines (sub)paragraphs to behave more like sections
\ifx\paragraph\undefined\else
\let\oldparagraph\paragraph
\renewcommand{\paragraph}[1]{\oldparagraph{#1}\mbox{}}
\fi
\ifx\subparagraph\undefined\else
\let\oldsubparagraph\subparagraph
\renewcommand{\subparagraph}[1]{\oldsubparagraph{#1}\mbox{}}
\fi

% set default figure placement to htbp
\makeatletter
\def\fps@figure{htbp}
\makeatother


\title{Advanced IP}
\author{QueenField}
\date{}

\begin{document}
\maketitle

\begin{figure}
\centering
\includegraphics{../icon.jpg}
\caption{QueenField}
\end{figure}

\section{1. HARDWARE PLANNING PROCESS}\label{hardware-planning-process}

\emph{The hardware plans describe the processes, procedures, methods,
and standards to be used for the hardware certification, design,
validation, verification, process assurance and configuration control.}

\subsection{1.1. Plan for Hardware Aspects of
Certification}\label{plan-for-hardware-aspects-of-certification}

\emph{The PHAC defines the processes, procedures, methods and standards
to be used to achieve the objectives of this document and obtain
certification authority approval for certification of the system
containing hardware items. The PHAC, once approved, represents an
agreement between the certification applicant and the certification
authority on the processes and activities to be conducted and the
resultant evidence to be produced to satisfy the hardware aspects of
certification. The PHAC may be part of another plan, such as the
airborne system certification plan.}

\subsection{1.2. Hardware Design Plan}\label{hardware-design-plan}

\emph{The hardware design plan describes the procedures, methods and
standards to be applied and the processes and activities to be conducted
for the design of the hardware item. This plan may be included in the
PHAC and may reference design policies and standards to be applied.}

\subsection{1.3. Hardware Validation
Plan}\label{hardware-validation-plan}

\emph{The validation plan describes the procedures, methods and
standards to be applied and the processes and activities to be conducted
for the validation of the hardware item derived requirements to achieve
the validation objectives of this document. This plan may be included in
the PHAC and may reference validation standards to be applied.}

\subsection{1.4. Hardware Verification
Plan}\label{hardware-verification-plan}

\emph{The verification plan describes the procedures, methods and
standards to be applied and the processes and activities to be conducted
for the verification of the hardware items to achieve the verification
objectives of this document. This plan may be included in the PHAC and
may reference verification policies and standards to be applied.}

\subsection{1.5. Hardware Configuration Management
Plan}\label{hardware-configuration-management-plan}

\emph{The hardware configuration management plan describes the policies,
procedures, standards and methods to be used to satisfy the
configuration management objectives of this document.}

\subsection{1.6. Hardware Process Assurance
Plan}\label{hardware-process-assurance-plan}

\emph{The hardware process assurance plan describes the procedures,
methods and standards to be applied and the processes and activities to
be conducted for achieving the process assurance objectives of this
document.}

\section{2. HARDWARE DESIGN PROCESS}\label{hardware-design-process}

\emph{The hardware design processes produce a hardware item that
fulfills the requirements allocated to hardware from the system
requirements. These are Requirements Capture, Conceptual Design,
Detailed Design, Implementation and Production Transition. These design
processes may be applied at any hierarchical level of the hardware item,
such as LRUs, circuit board assemblies and ASICs/PLDs. The following
sections describe each process, its objectives and the related
activities that should be addressed to reduce the probability of design
and implementation errors that affect safety. It is important that each
of these processes is planned and the details recorded in a hardware
design plan.}

\subsection{2.1. Requirements Capture
Process}\label{requirements-capture-process}

\emph{The requirements capture process identifies and records the
hardware item requirements. This includes those derived requirements
imposed by the proposed hardware item architecture, choice of
technology, the basic and optional functionality, environmental, and
performance requirements as well as the requirements imposed by the
system safety assessment. This process may be iterative since additional
requirements may become known during design.}

\subsection{2.2. Conceptual Design
Process}\label{conceptual-design-process}

\emph{The conceptual design process produces a high-level design concept
that may be assessed to determine the potential for the resulting design
implementation to meet the requirements. This may be accomplished using
such items as functional block diagrams, design and architecture
descriptions, circuit card assembly outlines, and chassis sketches.}

\subsection{2.3. Detailed Design Process}\label{detailed-design-process}

\emph{The detailed design process produces detailed design data using
the hardware item requirements and conceptual design data as the basis
for the detailed design.}

\subsection{2.4. Implementation Process}\label{implementation-process}

\emph{The implementation process uses the detailed design data to
produce the hardware item that is an input to the testing activity.}

\subsection{2.5. Production Transition}\label{production-transition}

\emph{In this process, manufacturing data, test facilities and general
resources should be examined to ensure availa bility and suitability for
production. The production transition process uses the outputs from the
implementation and verification processes to move the product into
production.}

\subsection{2.6. Acceptance Test}\label{acceptance-test}

\emph{An acceptance test demonstrates that the manufactured, modified or
repaired product performs in compliance with the key attributes of the
unit on which certification is based. These key attributes are chosen
using engineering judgement and are indicative that the product is
capable of meeting the requirements to which the unit was developed.}

\subsection{2.7. Series Production}\label{series-production}

\emph{This process is not within the scope of this document, but
elements impacting design assurance are briefly described to complete
the life cycle.}

\emph{This process reproduces the hardware item on a routine basis that
complies with the production data and requirements.}

\section{3. VALIDATION AND VERIFICATION
PROCESS}\label{validation-and-verification-process}

\emph{This section describes the validation process and the verification
process. The validation process provides assurance that the hardware
item derived requirements are correct and complete with respect to
system requirements allocated to the hardware item. The verification
process provides assurance that the hardware item implementation meets
all of the hardware requirements, including derived requirements.}

\subsection{3.1. Validation Process}\label{validation-process}

\emph{The validation process discussed here is intended to ensure that
the derived requirements are correct and complete with respect to the
system requirements allocated to the hardware item through the use of a
combination of objective and subjective processes. Validation may be
conducted before or after the hardware item is available, however,
validation is typically conducted throughout the design life cycle.}

\subsection{3.2. Verification Process}\label{verification-process}

\emph{The verification process provides assurance that the hardware item
implementation meets the requirements. Verification consists of reviews,
analyses and tests applied as defined in the verification plan. The
verification process should include an assessment of the results.}

\subsection{3.3. Validation and Verification
Methods}\label{validation-and-verification-methods}

\emph{This section describes some methods that may be applicable to both
validation and verification.}

\section{4. CONFIGURATION MANAGEMENT
PROCESS}\label{configuration-management-process}

\emph{The configuration management process is intended to provide the
ability to consistently replicate the configuration item, regenerate the
information if necessary and modify the configuration item in a
controlled fashion if modification is necessary. This section describes
the objectives for hardware configuration management and activities that
support those objectives.}

\subsection{4.1. Configuration Management
Objectives}\label{configuration-management-objectives}

\subsection{4.2. Configuration Management
Activities}\label{configuration-management-activities}

\subsection{4.3. Data Control Categories}\label{data-control-categories}

\section{5. PROCESS ASSURANCE}\label{process-assurance}

\emph{Process assurance ensures that the life cycle process objectives
are met and activities have been completed as outlined in plans or that
deviations have been addressed. This section describes the objectives
for process assurance and the activities that support those objectives.
There is no intent to impose specific organizational structures.}

\subsection{5.1. Process Assurance
Objectives}\label{process-assurance-objectives}

\subsection{5.2. Process Assurance
Activities}\label{process-assurance-activities}

\section{6. CERTIFICATION LIAISON
PROCESS}\label{certification-liaison-process}

\emph{The purpose of the certification liaison process is to establish
communication and understanding between the applicant and the
certification authority throughout the hardware design life cycle to
assist in the certification process. In addition, liaison activities may
include design approach presentation for timely approval, negotiations
concerning the means of compliance with the certification basis,
approval of design approach, means of data approval, and any required
certification authority reviews and witnessing of tests.}

\subsection{6.1. Means of Compliance and
Planning}\label{means-of-compliance-and-planning}

\emph{The applicant proposes a means of compliance for hardware. The
PHAC defines the proposed means of compliance.}

\subsection{6.2. Compliance
Substantiation}\label{compliance-substantiation}

\emph{The applicant provides evidence that the hardware design life
cycle processes have satisfied the hardware plans. Certification
authority reviews may take place at the applicant's facilities or
applicant's supplier's facilities. The applicant arranges these reviews
and makes hardware design life cycle data available as needed.}

\section{7. HARDWARE DESIGN LIFECYCLE
DATA}\label{hardware-design-lifecycle-data}

\emph{This section describes the hardware design life cycle data items
that may be produced during the hardware design life cycle for providing
evidence of design assurance and compliance with certification
requirements. The scope, amount and detail of the life cycle data needed
by the certification authorities as design assurance evidence will vary
depending on a number of factors. These factors include the applicable
certification authority requirements for the airborne system, the
assigned design assurance levels, the complexity and the service
experience of the hardware. Details of the design assurance evidence
should be identified, recorded in the PHAC and agreed to with the
certification authorities.}

\subsection{7.1. Hardware Plans}\label{hardware-plans}

\subsubsection{7.1.1. Plan for Hardware Aspects of
Certification}\label{plan-for-hardware-aspects-of-certification-1}

\subsubsection{7.1.2. Hardware Design
Plan}\label{hardware-design-plan-1}

\subsubsection{7.1.3. Hardware Validation
Plan}\label{hardware-validation-plan-1}

\subsubsection{7.1.4. Hardware Verification
Plan}\label{hardware-verification-plan-1}

\subsubsection{7.1.5. Hardware Configuration Management
Plan}\label{hardware-configuration-management-plan-1}

\subsubsection{7.1.6. Hardware Process Assurance
Plan}\label{hardware-process-assurance-plan-1}

\subsection{7.2. Hardware Design Standards and
Guidance}\label{hardware-design-standards-and-guidance}

\subsubsection{7.2.1. Requirements
Standards}\label{requirements-standards}

\subsubsection{7.2.2. Hardware Design
Standards}\label{hardware-design-standards}

\subsubsection{7.2.3. Validation and Verification
Standards}\label{validation-and-verification-standards}

\subsubsection{7.2.4. Hardware Archive
Standards}\label{hardware-archive-standards}

\subsection{7.3. Hardware Design Data}\label{hardware-design-data}

\subsubsection{7.3.1. Hardware
Requirements}\label{hardware-requirements}

\subsubsection{7.3.2. Hardware Design Representation
Data}\label{hardware-design-representation-data}

\paragraph{7.3.2.1. Conceptual Design
Data}\label{conceptual-design-data}

\paragraph{7.3.2.2. Detailed Design Data}\label{detailed-design-data}

\subparagraph{7.3.2.2.1. Top-Level Drawing}\label{top-level-drawing}

\subparagraph{7.3.2.2.2. Assembly Drawings}\label{assembly-drawings}

\subparagraph{7.3.2.2.3. Installation Control
Drawings}\label{installation-control-drawings}

\subparagraph{7.3.2.2.4. Hardware/Software Interface
Data}\label{hardwaresoftware-interface-data}

\subsection{7.4. Validation and Verification
Data}\label{validation-and-verification-data}

\subsubsection{7.4.1. Traceability Data}\label{traceability-data}

\subsubsection{7.4.2. Review and Analysis
Procedures}\label{review-and-analysis-procedures}

\subsubsection{7.4.3. Review and Analysis
Results}\label{review-and-analysis-results}

\subsubsection{7.4.4. Test Procedures}\label{test-procedures}

\subsubsection{7.4.5. Test Results}\label{test-results}

\subsection{7.5. Hardware Acceptance Test
Criteria}\label{hardware-acceptance-test-criteria}

\subsection{7.6. Problem Reports}\label{problem-reports}

\subsection{7.7. Hardware Configuration Management
Records}\label{hardware-configuration-management-records}

\subsection{7.8. Hardware Process Assurance
Records}\label{hardware-process-assurance-records}

\subsection{7.9. Hardware Accomplishment
Summary}\label{hardware-accomplishment-summary}

\section{8. ADDITIONAL CONSIDERATIONS}\label{additional-considerations}

\emph{This section provides guidance on additional considerations of
design assurance that are not covered in the previous sections. Any use
of additional considerations should be agreed with the certification
authority.}

\subsection{8.1. Use of Previously Developed
Hardware}\label{use-of-previously-developed-hardware}

\emph{This section discusses the issues associated with the use of
previously developed hardware. Guidance includes the assessment of
modifications to the hardware, to the aircraft installation, to the
application environment, or to the design environment and upgrading
design baselines. Guidance for COTS component usage, a special case of
previously developed hardware, is covered. Configuration Management and
Process Assurance considerations should also be addressed for each use
of previously developed hardware.}

\subsection{8.2. Commercial Components
Usage}\label{commercial-components-usage}

\emph{COTS components are used extensively in hardware designs and
typically the COTS components design data is not available for review.
The certification process does not specifically address individual
components, modules, or subassemblies, as these are covered as part of
the specific aircraft function being certified. As such, the use of COTS
components will be verified through the overall design process,
including the supporting processes, as defined in this document. The use
of an electronic component management process, in conjunction with the
design process, provides the basis for COTS components usage.}

\subsection{8.3. Product Service
Experience}\label{product-service-experience}

\emph{Service experience may be used to substantiate design assurance
for previously developed hardware and for COTS components. Service
experience relates to data collected from any previous or current usage
of the component. Data from non-airborne applications is not excluded.}

\subsection{8.4. Tool Assessment and
Qualification}\label{tool-assessment-and-qualification}

\emph{Tools, both hardware and software, will normally be used during
hardware design and verification. When design tools are used to generate
the hardware item or the hardware design, an error in the tool could
introduce an error in the hardware item. When verification tools are
used to verify the hardware item, an error in the tool may cause the
tool to fail to detect an error in the hardware item or hardware design.
Prior to the use of a tool, a tool assessment should be performed. The
results of this assessment and, if necessary, tool qualification should
be recorded and maintained.}

\end{document}
